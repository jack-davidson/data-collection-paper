\documentclass[12pt]{report}
\usepackage[utf8]{inputenc}
\usepackage{url}

\renewcommand{\bibname}{Works Cited}
\linespread{2}

\title{Exploring The Dangers of Data Collection}
\author{Jack Davidson}
\begin{document}
\maketitle

\par
	Data collection is rampant in our society and significant quantities of data are
	collected about the people in a multitude of varying mediums. The government
	infringes on the privacy of the people by harvesting vast amounts of data from
	the general population while school districts employ educational software that
	stores and collects every online movement that students take.

% The harms of data collection through the government.
\par
	In a hearing held in 2017 to review the usage of data collection for use with 
	facial recognition in law enforcement, over half of adults were found to be
	part of government facial recognition databases of which they may not even
	know they are a part \cite{reform}.

	The large amount of data collected on citizens "gives law enforcement a power
	they've never had before" \cite{reform}. Information is so easily and readily
	available to law enforcement. In the protests following the murder of
	Freddie Gray, Baltimore police "used [facial] recognition on social media
	photos to identify [protestors]. These targeted protesters were then arrested
	for unrelated causes \cite{spivak}. Many peaceful protesters were monitored
	without reason \cite{reform}.

	Covert collection of data by the government is unjust because citizens should
	be able to trust that the government is acting in their best interest. The
	government should only monitor citizens when there is a just cause. Monitoring
	the general population without just cause is a significant breach of privacy.

% The harms of data collection through schooling software.
\par
	Go Guardian, a suite of web-based student monitoring and proctoring programs,
	collects data about our most precious segment of the population,
	children. This harmful student monitoring program is used at many
	districts across the United States including The Austin Independent
	School District, The School District of which Kealing Middle School is a part.
	School-issued Google Chromebooks have the proctoring software permanently
	installed. Teachers, administrators, and Go Guardian itself can remotely
	access and control the computers of students. Many of those AISD Google
	Chromebooks are the first personal computers that such students have ever had
	access to.

	With web browsers being large, complex programs, often having various
	privilege escalation vulnerabilities, Go Guardian could have the ability
	to access the rest of your computer including running programs,
	accessing the network, and reading or writing your personal files. If a
	student was forced to install Go Guardian on their personal computer, sensitive
	information could be accessed by Go Guardian such as banking information, legal
	information, and other sensitive documents.

	Go Guardian "[can] share information with our service providers that [\ldots]
	support our Offerings" \cite{policy}. The data Go Guardian collect and they
	can use it for any purpose such as selling or sharing it with other companies.
	Go Guardian receives money to collect data about students. School districts
	put forward vast amounts of resources for these malicious services. Go Guardian
	license rates cost around \$12 per year for each device. In a school district
	such as Austin Independent School District with a population of around 80,890
	students, to pay for Go Guardian licenses for only \textbf{one fourth of the
	students} in the District, it would cost \$242,670 for \textbf{one year} of
	the service \cite{population}. \$242,670 is a large cost for such a harmful
	service.

	The fact that Go Guardian is charging for a service \emph{and} profiting
	from the data collection is abusive to the district and abusive to the student.
	Invasive educational software products like Go Guardian are a strain on school
	districts. The resources of school districts like AISD could be much better
	utilized by raising teacher salaries or providing more resources to students.
	Go Guardian is both a strain on privacy and a strain on resources.

% The harms of data collection through technology
% services such as Amazon, Netflix, etc.
\par
	Private companies such as Amazon, Netflix, and Facebook all collect
	personal data about you to strengthen their very accurate yet abusive
	recommendation system.

	In exchange for services like Google, Facebook, and Reddit, the consumer
	returns their precious personal data. No private company is going to operate
	for free \cite{hill}. Technology companies can do any arbitrary thing they
	please with the consumer's data such as selling or giving as part of an
	agreement to third parties, using it for advertisement, or using it to tune
	their internal algorithms. Technology companies like Google, Facebook, and
	Reddit form a parasitic symbiosis between the producer and the consumer.

\par
	Both governmental and technological organizations collect personal data
	and control society with it. With data being collected through medium it
	can be hard to know what software we can use without worrying about
	being spied on, cornered, and abused. Luckily, there are ways we can
	circumvent such abuse. We can refuse to use these services even if it may be
	inconvenient and we can use only software developed by the \emph{community}
	and \emph{controlled} by the \emph{users}.

\newpage

\begin{thebibliography}{1}
	\bibitem{reform}
		``Facial Recognition Technology (Part 1): Its Impact on our Civil Rights
		and Liberties''.
		House Committee on Oversight and Reform,
		2019,
		\url{https://oversight.house.gov/legislation/hearings/facial-recognition-technology-part-1-its-impact-on-our-civil-rights-and}.
	
	\bibitem{population}
		``AISD District Demographics''.
		Austin Independent School District,
		2020,
		\url{https://www.austinisd.org/planning-asset-management/district-demographics}.

	\bibitem{policy}
		``Product Privacy Policy''.
		GoGuardian,
		Liminex Corporation,
		2020,
		\url{https://www.goguardian.com/product-privacy/}.

	\bibitem{hill}
		Hill, Steven.
		``Should Big Tech own our Personal Data''.
		Wired,
		Condé Nast,
		2019,
		\url{https://www.wired.com/story/should-big-tech-own-our-personal-data/}.
	
	\bibitem{spivak}
		Spivak, Jameson.
		``Maryland’s face recognition system is one of the most invasive in the nation | COMMENTARY''.
		The Baltimore Sun,
		Tribune Publishing,
		2020,
		\url{https://www.baltimoresun.com/opinion/op-ed/bs-ed-op-0310-face-recognition-20200309-hg6jkfav2fdz3ccs55bvqjtnmu-story.html}.
\end{thebibliography}

\end{document}
